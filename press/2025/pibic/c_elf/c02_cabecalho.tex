\section{Cabeçalho ELF (ELF Header)}\label{sec:elf_header}

O cabeçalho ELF é a estrutura inicial e mais fundamental de um arquivo ELF, fornecendo informações essenciais sobre o formato do arquivo, sua arquitetura alvo e pontos de entrada para processamento adicional. Este cabeçalho serve como um mapa que permite ao sistema operacional e outras ferramentas interpretarem corretamente o conteúdo do arquivo.

\subsection{Estrutura do Cabeçalho ELF}\label{subsec:elf_header_structure}

O cabeçalho ELF é uma estrutura de dados que ocupa os primeiros bytes do arquivo. Na especificação TIS, conforme mencionado na \secref{sec:tis_standard}, o cabeçalho é definido por uma estrutura chamada \texttt{Elf32\_Ehdr} (para arquivos ELF de 32 bits) ou \texttt{Elf64\_Ehdr} (para arquivos ELF de 64 bits).

\begin{lstlisting}[language=C, caption={Estrutura do cabeçalho ELF de 64 bits}, label={lst:elf64_header}]
typedef struct {
    unsigned char e_ident[EI_NIDENT]; /* Identificacao do arquivo ELF */
    Elf64_Half    e_type;             /* Tipo de objeto */
    Elf64_Half    e_machine;          /* Arquitetura necessaria */
    Elf64_Word    e_version;          /* Versao do objeto */
    Elf64_Addr    e_entry;            /* Endereco de entrada virtual */
    Elf64_Off     e_phoff;            /* Offset da tabela de program header */
    Elf64_Off     e_shoff;            /* Offset da tabela de section header */
    Elf64_Word    e_flags;            /* Flags especificas do processador */
    Elf64_Half    e_ehsize;           /* Tamanho do ELF header */
    Elf64_Half    e_phentsize;        /* Tamanho de uma entrada da program header table */
    Elf64_Half    e_phnum;            /* Numero de entradas na program header table */
    Elf64_Half    e_shentsize;        /* Tamanho de uma entrada da section header table */
    Elf64_Half    e_shnum;            /* Numero de entradas na section header table */
    Elf64_Half    e_shstrndx;         /* Indice da tabela de secoes que contem nomes de secoes */
} Elf64_Ehdr;
\end{lstlisting}

\subsection{Campo de Identificação (e\_ident)}\label{subsec:e_ident}

Os primeiros bytes do cabeçalho ELF, o array \texttt{e\_ident}, são particularmente importantes e contêm informações para identificação do arquivo:

\begin{table}[ht]
    \centering
    \caption{Campos do array e\_ident}
    \label{tab:e_ident_fields}
    \begin{tabular}{|l|c|l|}
        \hline
        \textbf{Índice} & \textbf{Nome} & \textbf{Descrição} \\
        \hline
        0-3 & EI\_MAG0 até EI\_MAG3 & "Magic number": 0x7F, 'E', 'L', 'F' \\
        4 & EI\_CLASS & Classe de arquivo (32 ou 64 bits) \\
        5 & EI\_DATA & Endianness dos dados (little/big-endian) \\
        6 & EI\_VERSION & Versão do ELF (geralmente 1) \\
        7 & EI\_OSABI & ABI do sistema operacional alvo \\
        8 & EI\_ABIVERSION & Versão da ABI específica \\
        9-15 & EI\_PAD & Bytes reservados, preenchidos com zeros \\
        \hline
    \end{tabular}
\end{table}

O "Magic Number" no início do arquivo (0x7F, 'E', 'L', 'F') é especialmente importante, pois permite que sistemas identifiquem rapidamente se um arquivo é do formato ELF ou não.

\subsection{Tipo de Arquivo (e\_type)}\label{subsec:e_type}

O campo \texttt{e\_type} especifica o tipo de arquivo ELF. Os valores mais comuns são:

\begin{itemize}
    \item \texttt{ET\_NONE} (0): Tipo não especificado
    \item \texttt{ET\_REL} (1): Arquivo relocável (objeto)
    \item \texttt{ET\_EXEC} (2): Arquivo executável
    \item \texttt{ET\_DYN} (3): Objeto compartilhado (biblioteca compartilhada)
    \item \texttt{ET\_CORE} (4): Arquivo core (dump de memória)
\end{itemize}

Esta classificação afeta diretamente como o arquivo será tratado pelo sistema operacional, como veremos em \chapref{chap:code} quando discutirmos o carregamento de arquivos ELF.

\subsection{Arquitetura Alvo (e\_machine)}\label{subsec:e_machine}

O campo \texttt{e\_machine} especifica a arquitetura para a qual o arquivo foi compilado. Alguns valores comuns incluem:

\begin{itemize}
    \item \texttt{EM\_386} (3): Intel 80386
    \item \texttt{EM\_X86\_64} (62): AMD x86-64
    \item \texttt{EM\_ARM} (40): ARM
    \item \texttt{EM\_AARCH64} (183): ARM 64-bits (AArch64)
    \item \texttt{EM\_RISCV} (243): RISC-V
\end{itemize}

Este campo é crítico para o sistema operacional determinar se o binário é compatível com o hardware em que está sendo executado. Como discutido na \secref{subsec:isa}, o ISA do processador deve corresponder àquele para o qual o binário foi compilado.

\begin{figure}[ht]
    \centering
    \rule{10cm}{6cm} % Placeholder para um diagrama real
    \caption{Estrutura do cabeçalho ELF e sua relação com o resto do arquivo}
    \label{fig:elf_header_structure}
\end{figure}

\subsection{Pontos de Entrada e Tabelas (e\_entry, e\_phoff, e\_shoff)}\label{subsec:entry_points}

Os campos \texttt{e\_entry}, \texttt{e\_phoff}, e \texttt{e\_shoff} são essenciais para o carregamento e execução do arquivo:

\begin{itemize}
    \item \texttt{e\_entry}: Endereço virtual do ponto de entrada do programa, onde a execução deve começar.
    \item \texttt{e\_phoff}: Offset, em bytes, do início da tabela de cabeçalhos de programa (Program Header Table).
    \item \texttt{e\_shoff}: Offset, em bytes, do início da tabela de cabeçalhos de seção (Section Header Table).
\end{itemize}

Como veremos nas seções \secref{sec:elf_segments} e \secref{sec:elf_sections}, estas tabelas fornecem informações detalhadas sobre como o arquivo deve ser carregado na memória e como suas partes estão organizadas.

\subsection{Contagens e Tamanhos (e\_phnum, e\_shnum, etc.)}\label{subsec:counts_sizes}

Os campos restantes fornecem informações sobre o número e tamanho das entradas nas tabelas de cabeçalho do programa e de seção:

\begin{itemize}
    \item \texttt{e\_ehsize}: Tamanho do cabeçalho ELF em bytes.
    \item \texttt{e\_phentsize}: Tamanho de cada entrada na Program Header Table.
    \item \texttt{e\_phnum}: Número de entradas na Program Header Table.
    \item \texttt{e\_shentsize}: Tamanho de cada entrada na Section Header Table.
    \item \texttt{e\_shnum}: Número de entradas na Section Header Table.
    \item \texttt{e\_shstrndx}: Índice da seção na Section Header Table que contém os nomes das seções.
\end{itemize}

Estes campos permitem que o sistema operacional ou outras ferramentas naveguem corretamente pelo arquivo e localizem informações específicas.

\subsection{Implementação Prática: Análise de um Cabeçalho ELF}\label{subsec:elf_header_analysis}

Na prática, podemos analisar o cabeçalho ELF usando ferramentas como \texttt{readelf}:

\begin{lstlisting}[language=bash, caption={Exemplo de uso do readelf para analisar o cabeçalho ELF}, label={lst:readelf_example}]
$ readelf -h /bin/ls
ELF Header:
  Magic:   7f 45 4c 46 02 01 01 00 00 00 00 00 00 00 00 00 
  Class:                             ELF64
  Data:                              2's complement, little endian
  Version:                           1 (current)
  OS/ABI:                            UNIX - System V
  ABI Version:                       0
  Type:                              DYN (Shared object file)
  Machine:                           Advanced Micro Devices X86-64
  Version:                           0x1
  Entry point address:               0x6050
  Start of program headers:          64 (bytes into file)
  Start of section headers:          137240 (bytes into file)
  Flags:                             0x0
  Size of this header:               64 (bytes)
  Size of program headers:           56 (bytes)
  Number of program headers:         13
  Size of section headers:           64 (bytes)
  Number of section headers:         31
  Section header string table index: 30
\end{lstlisting}

No \chapref{chap:code}, veremos como implementar nossa própria ferramenta para analisar cabeçalhos ELF e explorar sua estrutura interna.

\subsection{Considerações de Segurança}\label{subsec:elf_header_security}

O cabeçalho ELF é frequentemente alvo de manipulações em ataques de segurança:

\begin{itemize}
    \item Alterações maliciosas no campo \texttt{e\_entry} podem redirecionar a execução para código malicioso.
    \item Modificações nos campos de offset podem causar interpretação incorreta do arquivo.
    \item Ajustes nos valores de contagem podem levar a estouro de buffer ou outros problemas.
\end{itemize}

Estas considerações são relevantes para as aplicações de segurança discutidas no \chapref{chap:future}, particularmente na \secref{sec:antivirus} sobre detecção de malware.

\subsection{Relação com Outras Partes do ELF}\label{subsec:elf_header_relation}

O cabeçalho ELF funciona como ponto de entrada para todas as outras estruturas de dados no arquivo:

\begin{itemize}
    \item Aponta para a tabela de cabeçalhos de programa, que descreve os segmentos para carregamento (ver \secref{sec:elf_segments}).
    \item Aponta para a tabela de cabeçalhos de seção, que detalha as seções do arquivo (ver \secref{sec:elf_sections}).
    \item Define o ponto de entrada para execução, essencial para o carregador do sistema operacional.
\end{itemize}

Esta estrutura hierárquica permite navegação eficiente pelo arquivo e facilita o carregamento de seus componentes na memória, como veremos nas próximas seções.