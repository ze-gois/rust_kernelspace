\section{Seções ELF (Section Headers)}\label{sec:elf_sections}

As seções são as unidades fundamentais de organização em arquivos ELF, fornecendo uma visão detalhada do conteúdo do arquivo para ferramentas de desenvolvimento. Enquanto os segmentos, discutidos na \secref{sec:elf_segments}, são orientados à execução, as seções são orientadas ao processo de vinculação (linking) e manipulação do código.

\subsection{Conceito e Propósito}\label{subsec:section_purpose}

As seções dividem o arquivo ELF em unidades lógicas baseadas no tipo de conteúdo e propósito:

\begin{itemize}
    \item Organizam o código e dados em categorias distintas
    \item Permitem o acesso seletivo a diferentes partes do arquivo
    \item Facilitam a manipulação por ferramentas como compiladores, linkers e depuradores
    \item Fornecem metadados necessários para processamento específico
\end{itemize}

As seções são particularmente importantes durante o desenvolvimento e para ferramentas de análise, como veremos em maior detalhe no \chapref{chap:code}.

\subsection{Estrutura do Section Header}\label{subsec:section_header_structure}

Cada seção em um arquivo ELF é descrita por uma entrada na Section Header Table. Para ELF de 64 bits, essa entrada é definida pela estrutura \texttt{Elf64\_Shdr}:

\begin{lstlisting}[language=C, caption={Estrutura do Section Header de 64 bits}, label={lst:elf64_shdr}]
typedef struct {
    Elf64_Word    sh_name;      /* Índice para o nome da seção */
    Elf64_Word    sh_type;      /* Tipo da seção */
    Elf64_Xword   sh_flags;     /* Atributos da seção */
    Elf64_Addr    sh_addr;      /* Endereço virtual, se carregável */
    Elf64_Off     sh_offset;    /* Offset no arquivo */
    Elf64_Xword   sh_size;      /* Tamanho da seção em bytes */
    Elf64_Word    sh_link;      /* Link para outra seção */
    Elf64_Word    sh_info;      /* Informações adicionais */
    Elf64_Xword   sh_addralign; /* Alinhamento necessário */
    Elf64_Xword   sh_entsize;   /* Tamanho da entrada, se tabela */
} Elf64_Shdr;
\end{lstlisting}

\subsection{Tipos de Seção}\label{subsec:section_types}

O campo \texttt{sh\_type} especifica o tipo da seção, determinando seu conteúdo e propósito:

\begin{itemize}
    \item \texttt{SHT\_NULL} (0): Seção inativa ou marcador
    \item \texttt{SHT\_PROGBITS} (1): Dados definidos pelo programa
    \item \texttt{SHT\_SYMTAB} (2): Tabela de símbolos
    \item \texttt{SHT\_STRTAB} (3): Tabela de strings
    \item \texttt{SHT\_RELA} (4): Entradas de relocação com adendos explícitos
    \item \texttt{SHT\_HASH} (5): Tabela de hash de símbolos
    \item \texttt{SHT\_DYNAMIC} (6): Informações para vinculação dinâmica
    \item \texttt{SHT\_NOTE} (7): Informações de marcação
    \item \texttt{SHT\_NOBITS} (8): Ocupa espaço mas não tem dados no arquivo (ex: BSS)
    \item \texttt{SHT\_REL} (9): Entradas de relocação sem adendos explícitos
    \item \texttt{SHT\_DYNSYM} (11): Tabela mínima de símbolos para ligação dinâmica
\end{itemize}

\subsection{Flags de Seção}\label{subsec:section_flags}

O campo \texttt{sh\_flags} define atributos para a seção:

\begin{itemize}
    \item \texttt{SHF\_WRITE} (0x1): Seção contém dados graváveis durante a execução
    \item \texttt{SHF\_ALLOC} (0x2): Seção ocupa memória durante a execução
    \item \texttt{SHF\_EXECINSTR} (0x4): Seção contém código executável
    \item \texttt{SHF\_MERGE} (0x10): Dados podem ser mesclados para economizar espaço
    \item \texttt{SHF\_STRINGS} (0x20): Seção contém strings terminadas em nulo
    \item \texttt{SHF\_INFO\_LINK} (0x40): Campo sh\_info contém índice de seção
    \item \texttt{SHF\_TLS} (0x400): Seção contém dados de Thread-Local Storage
\end{itemize}

Estas flags podem ser combinadas para definir múltiplos atributos.

\begin{figure}[ht]
    \centering
    \rule{10cm}{6cm} % Placeholder para um diagrama real
    \caption{Organização das seções em um arquivo ELF típico}
    \label{fig:section_organization}
\end{figure}

\subsection{Seções Especiais}\label{subsec:special_sections}

Certas seções têm significados específicos e padronizados:

\begin{table}[ht]
    \centering
    \caption{Seções comuns em arquivos ELF}
    \label{tab:common_sections}
    \begin{tabular}{|l|l|l|}
        \hline
        \textbf{Nome} & \textbf{Tipo} & \textbf{Descrição} \\
        \hline
        .text & SHT\_PROGBITS & Código executável do programa \\
        .data & SHT\_PROGBITS & Dados inicializados (variáveis globais) \\
        .bss & SHT\_NOBITS & Dados não inicializados (ocupa espaço apenas na memória) \\
        .rodata & SHT\_PROGBITS & Dados somente para leitura (constantes) \\
        .symtab & SHT\_SYMTAB & Tabela de símbolos completa \\
        .strtab & SHT\_STRTAB & Tabela de strings para nomes de símbolos \\
        .shstrtab & SHT\_STRTAB & Tabela de strings para nomes de seções \\
        .dynamic & SHT\_DYNAMIC & Informações para vinculação dinâmica \\
        .got & SHT\_PROGBITS & Global Offset Table (para código com posição independente) \\
        .plt & SHT\_PROGBITS & Procedure Linkage Table (para chamadas em bibliotecas compartilhadas) \\
        .init & SHT\_PROGBITS & Código de inicialização executado antes da função main() \\
        .fini & SHT\_PROGBITS & Código de finalização executado após o término do programa \\
        \hline
    \end{tabular}
\end{table}

\subsection{Tabela de Strings da Seção}\label{subsec:shstrtab}

Os nomes das seções não são armazenados diretamente nos cabeçalhos de seção. Em vez disso, o campo \texttt{sh\_name} contém um índice para a tabela de strings \texttt{.shstrtab}. Esta tabela armazena todos os nomes de seção como strings terminadas em nulo, concatenadas em um único bloco.

O índice para esta tabela de strings é especificado no campo \texttt{e\_shstrndx} do cabeçalho ELF, como mencionado na \secref{sec:elf_header}.

\subsection{Relação com Segmentos}\label{subsec:sections_to_segments}

Múltiplas seções podem ser combinadas em um único segmento para carregamento na memória:

\begin{itemize}
    \item Seções são a visão lógica, orientada à vinculação
    \item Segmentos são a visão física, orientada à execução
    \item O mapeamento entre seções e segmentos é definido pela posição e atributos
\end{itemize}

Por exemplo, as seções \texttt{.text}, \texttt{.rodata} e outras seções somente leitura seriam tipicamente mapeadas para um segmento \texttt{PT\_LOAD} com permissões R-X, enquanto \texttt{.data} e \texttt{.bss} seriam mapeadas para um segmento \texttt{PT\_LOAD} com permissões RW-.

\subsection{Análise de Seções com Ferramentas}\label{subsec:section_analysis}

As seções de um arquivo ELF podem ser analisadas utilizando ferramentas como \texttt{readelf}:

\begin{lstlisting}[language=bash, caption={Exemplo de uso do readelf para analisar seções}, label={lst:readelf_sections}]
$ readelf -S /bin/ls
There are 29 section headers, starting at offset 0x21768:

Section Headers:
  [Nr] Name              Type             Address           Offset
       Size              EntSize          Flags  Link  Info  Align
  [ 0]                   NULL             0000000000000000  00000000
       0000000000000000  0000000000000000           0     0     0
  [ 1] .interp           PROGBITS         0000000000000318  00000318
       000000000000001c  0000000000000000   A       0     0     1
  [ 2] .note.gnu.build-i NOTE             0000000000000338  00000338
       0000000000000024  0000000000000000   A       0     0     4
  ...
  [11] .text             PROGBITS         0000000000005000  00005000
       000000000000efc2  0000000000000000  AX       0     0     16
  ...
\end{lstlisting}

\subsection{Seções de Depuração}\label{subsec:debug_sections}

Arquivos ELF podem incluir seções especiais para informações de depuração:

\begin{itemize}
    \item \texttt{.debug\_info}: Informações gerais de depuração
    \item \texttt{.debug\_line}: Mapeamento entre código objeto e código fonte
    \item \texttt{.debug\_abbrev}: Abreviações para compressão de dados de depuração
    \item \texttt{.debug\_str}: Strings utilizadas nas informações de depuração
\end{itemize}

Estas seções seguem frequentemente o formato DWARF, um padrão para informações de depuração que complementa o ELF. Estas informações são cruciais para depuradores como GDB e para ferramentas de análise que veremos no \chapref{chap:code}.

\subsection{Uso de Seções em Análise Binária}\label{subsec:section_binary_analysis}

A análise de seções ELF é particularmente útil em:

\begin{itemize}
    \item Engenharia reversa: Identificação de código e dados
    \item Segurança: Detecção de seções suspeitas ou modificadas
    \item Otimização: Análise de tamanho e organização das seções
    \item Depuração: Correlação entre código binário e código fonte
\end{itemize}

Como será discutido no \chapref{chap:future}, estas aplicações são fundamentais para áreas como detecção de malware e patching de software.

\subsection{Extensões Específicas}\label{subsec:section_extensions}

Diferentes implementações podem adicionar seções específicas:

\begin{itemize}
    \item GNU: Seções como \texttt{.gnu.version}, \texttt{.gnu.hash}
    \item Linux: Seções específicas do kernel como \texttt{.modinfo}
    \item Compiladores: Seções para otimizações específicas
\end{itemize}

A flexibilidade do formato ELF permite estas extensões sem comprometer a compatibilidade básica, um dos motivos de sua ampla adoção em sistemas Unix-like.