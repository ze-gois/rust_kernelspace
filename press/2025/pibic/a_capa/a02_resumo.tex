\chapter*{Resumo}

Este trabalho apresenta um estudo detalhado sobre o formato de arquivo ELF (Executable and Linkable Format), amplamente utilizado em sistemas operacionais Unix-like como formato padrão para executáveis, bibliotecas compartilhadas, arquivos objeto e core dumps. A pesquisa abrange desde conceitos fundamentais de arquitetura de computadores e sistemas operacionais até uma análise minuciosa da estrutura interna dos arquivos ELF, incluindo cabeçalhos, segmentos e seções.

O trabalho também descreve a implementação de uma ferramenta para análise e manipulação de arquivos ELF, comparando-a com soluções existentes como elfutils. São discutidas aplicações práticas dessa ferramenta, como detecção de software malicioso e patching de binários.

Os resultados obtidos demonstram a importância do entendimento aprofundado do formato ELF para áreas como desenvolvimento de sistemas, segurança computacional e engenharia reversa de software. A ferramenta desenvolvida oferece uma alternativa moderna e flexível para trabalhar com arquivos ELF em ambientes de desenvolvimento e pesquisa.

\vspace{0.5cm}
\noindent
\textbf{Palavras-chave:} ELF, sistemas operacionais, formato de arquivo, análise binária, compiladores, linkers.